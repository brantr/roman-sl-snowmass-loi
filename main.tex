\documentclass[11pt]{article}
\usepackage{times}
\usepackage{geometry}
\geometry{letterpaper, portrait, margin=1in}
\usepackage[utf8]{inputenc}
\usepackage{enumitem,amssymb}
\usepackage{ragged2e}
\usepackage{graphicx}
\usepackage{comment}
\usepackage{multicol}
\usepackage[usenames]{xcolor} %used for font color              
\definecolor{xlinkcolor}{cmyk}{1,1,0,0}
\usepackage{url}
\usepackage[
 colorlinks=true,    % false: boxed links; true: colored links 
 linkcolor=xlinkcolor,     % color of internal links            
 citecolor=xlinkcolor,     % color of links to bibliography
 filecolor=xlinkcolor,  % color of file links 
 urlcolor=xlinkcolor,      % color of external link
 final=true
]{hyperref}
\usepackage[super,sort&compress]{natbib}
\usepackage{enumitem}
\setenumerate{itemsep=0mm}

\setlength{\parskip}{0.5em}

\bibliographystyle{naturemag-doi}


\begin{document}
\begin{raggedright} 
% part of template, but does not look good
\huge
Snowmass2021 - Letter of Interest \hfill \\[+1em]
\textit{Roman Space Telescope Strong Lensing Probes of Dark Matter Substructure} \hfill \\[+1em]
\end{raggedright}

\normalsize

\noindent {\large \bf Thematic Areas:}  (check all that apply $\square$/$\blacksquare$)

\noindent $\square$ (CF1) Dark Matter: Particle Like \\
\noindent $\square$ (CF2) Dark Matter: Wavelike  \\ 
\noindent $\blacksquare$ (CF3) Dark Matter: Cosmic Probes  \\
\noindent $\square$ (CF4) Dark Energy and Cosmic Acceleration: The Modern Universe \\
\noindent $\square$ (CF5) Dark Energy and Cosmic Acceleration: Cosmic Dawn and Before \\
\noindent $\square$ (CF6) Dark Energy and Cosmic Acceleration: Complementarity of Probes and New Facilities \\
\noindent $\square$ (CF7) Cosmic Probes of Fundamental Physics \\
\noindent $\square$ (Other) {\it [Please specify frontier/topical group]} \\

\noindent {\large \bf Contact Information:} (authors listed after the text)\\
Submitter Name/Institution: \\
Collaboration (optional): \\
Contact Email: \\

\noindent {\large \bf Abstract:} (must fit on this page)

\clearpage

% LOI text
\noindent {\it Insert your white paper text here (maximum of 2 pages including figures).}


The early growth of SMBHs discovered by WFIRST may strain our models for galaxy
formation, but WFIRST can test our picture for small-scale dark matter structure formation more
directly through gravitational lensing. The cold dark matter plus cosmological constant (LCDM)
paradigm forms one of the pillars of our models for the origin and evolution of cosmic structures.
While remarkably successful at matching observations on large scales, there have been persistent
observational challenges to the cold, collisionless dark matter expectations on dwarf-galaxy
scales. These “small-scale controversies” may simply stem from a poor understanding of the
baryonic processes involved in galaxy formation, or indicate more complex dark sector physics.

Detailed testing of the standard paradigm on small scales remains one of the most pressing issues
in cosmology. Numerical simulations in $\Lambda$CDM predict a rich spectrum of substructure in galaxy
halos. Small fluctuations in the galaxy-scale lensing potential caused by these substructures
should result in measurable “flux anomalies” in the magnifications of quadruply-lensed quasar
images \citep{metcalf2001a}. While discrepancies between the observed flux ratios and
those predicted by a smooth lens model may have been found in radio quasar lenses \citep{mao1998a,dalal2002a,metcalf2002a}, the small sample size ($\sim20$
lenses) of current samples limits our understanding. WFIRST HLS will revolutionize this field
by increasing the sample of quad lenses more than 10$\times$ \citep{oguri2010a}.
More recently, methods have been developed to use extended lensed sources, in addition to the
point-like AGN, to identify substructures through distortions of lensed images. Extended sources
are imaged into arcs and rings that probe a larger volume in the lens halo, increasing the number
of subhalos that can be sensed. They also can be drawn from larger populations of background
sources, typical galaxies rather than bright AGN. \cite[Vegetti \& Koopman (2009)][]{vegetti2009a} described a
method for identifying subtrsuctures with high-resolution optical/IR imaging of extended
galaxies, with \cite{vegetti2012a} reporting a detection of a dark substructure in a gravitational
lens at $z=0.9$. In related work, \cite{hezaveh2013a} described the use of (sub-)millimeter
spectral lines from dusty lensed galaxies to search for subhalos in the lenses. The wide survey
area of HLS, along with its combination of multi-band imaging and spectroscopy, will provide
many avenues for identifying large samples of lensed galaxies for such substructure searches.
The EXPO team will explore the substructure science achievable from the HLS and various GO
options. We will predict survey yields for lenses that will be suitable for all three substructure
probes. These counts will be synthesized into constraints on the substructure mass function,
factoring in the parameters of realistic followup campaigns.
\clearpage

\noindent %{\large \bf References:} (hyperlinks welcome)


\def\apj{\it{ApJ}}                  
\def\apjl{\it{ApJL}}
\def\mnras{\it{MNRAS}}
\def\nat{\it{Nature}}

\bibliography{references}


\vspace{4in}

\noindent {\large \bf Authors:}

Brant Robertson (UC Santa Cruz)





\end{document}
